\chapter{Systèmes cryptographiques et algorithmes de chiffrement}
	\section{Algorithmes symétriques et asymétriques}
		\subsection{Algorithme de chiffrement symétrique}
			Un algorithme de chiffrement est dit symétrique si un message chiffré avec cet algorithme peut être déchiffré en utilisant la clé qui a servi à le chiffrer. En général, le niveau de sécurité qu'offrent les algorithmes de ce type dépend de la complexité de la clé. C'est le type de chiffrement qui est utilisé la plupart du temps lorsqu'un client communique avec un serveur, car les algorithmes de chiffrement symétriques ont la particularité d'être - pour la plupart - rapides et efficaces. La grosse faiblesse de ces algorithmes est le fait que les deux parties communicantes doivent connaître la clé. Cela implique un échange de clés au départ de la communication. Pour cela, en pratique, un algorithme de chiffrement asymétrique est utilisé afin que personne d'autre que les deux parties communicantes ne puisse connaître la clé.
		\subsection{Algorithme de chiffrement asymétrique}
			Un algorithme de chiffrement est dit asymétrique si la clé ayant servi à chiffrer un message en utilisant cet algorithme n'est pas la même que la clé qui servira à le déchiffrer plus tard. En pratique, les deux parties s'échangent leurs clés publiques, mais conservent leurs clés privées. Les clés publiques servent à chiffrer les messages, et les clés privées servent à déchiffrer les messages. Ces algorithmes sont souvent lourds car ils impliquent la manipulation de nombres entiers très grands, ce qui se révèle être très coûteux en terme de temps processeur. C'est pour cela qu'en pratique, ces algorithmes sont utilisés pour échanger des clés de chiffrement pour des algorithmes de chiffrement symétriques, qui sont beaucoup plus rapides. L'avantage de ces algorithmes, c'est que c'est un des seuls moyens de transmettre un message en s'assurant qu'un tiers ne pourra pas le lire, même si l'échange de clés a lieu sous ses yeux. Les deux parties peuvent donc s'échanger leurs clés publiques sans crainte de voir la sécurité du message remise en question. De plus, ces algorithmes permettent également de signer un message, c'est-à-dire qu'en calculant une signature (résultat d'une fonction de hachage) du message, et en le chiffrant en utilisant la clé privée, tout le monde possédant la clé publique pourra s'assurer que le message provient bien du bon expéditeur. C'est d'ailleurs la technique employée afin d'identifier des serveurs Web, via des certificats délivrés par des autorités de confiance qui permettent de prouver la bonne foi de l'expéditeur de la page que nous sommes en train de visiter.
		\subsection{Algorithme de hachage}
			Les algorithmes dit de hachage ne sont pas des algorithmes de chiffrement à proprement parler, mais plutôt des algorithmes de cryptage, c'est-à-dire que le but d'un algorithme de hachage n'est pas de sécuriser la transmission d'un message, mais plutôt de rendre un message irrécupérable, mais d'être capable de l'identifier par sa signature. Nous pouvons ici prendre l'exemple de l'algorithme MD5, qui sert la plupart du temps à confirmer qu'un fichier téléchargé est complet et qu'il n'a pas subi d'erreurs de transmission. Pour cela, on compare la signature du fichier obtenue par l'algorithme MD5 du serveur et du client. Si les deux signatures sont égales, alors on peut affirmer qu'un fichier a une probabilité très grande de ne pas avoir subi d'erreurs. Ce genre d'algorithme peut aussi être utilisé afin de cacher des mots de passe dans une base de données. En sauvegardant uniquement le résultat de la fonction de hachage dans la base de données, on est capable de déterminer si le mot de passe entré est le bon ou pas, en comparant la signature enregistrée à la signature du mot de passe entré, si la fonction de hachage donne des résultats suffisamment variés afin que deux mots de passe ne donnent pas le même résultat par cette fonction. Ce procédé se trouvera forcément confronté au problème du paradoxe des anniversaires, c'est-à-dire que la probabilité que deux mots de passe aient la même signature dans la base de données si la fonction de hachage donne un nombre fini de résultats possibles croît de manière sigmoïdale en fonction du nombre de mots de passe dans la base de données et du nombre de résultats possibles de la fonction de hachage, c'est-à-dire que pour un petit nombre d'entrées la probabilité est très faible, puis elle augmente beaucoup, puis elle tend vers 1 (cette probabilité étant atteinte lorsque le nombre de mots de passe de la base de données est égal au nombre de résultats possibles de la fonction de hachage).
	\section{Système de Vigenère}
		\subsection{Définition}
			Le système de Vigenère est une méthode de chiffrement ayant une clé de longueur variable, inférieure ou égale à celle du message à transmettre. Lorsqu'on parle de système de Vigenère, on pense à l'utiliser dans le cadre de l'alphabet français, qui ne comporte que 26 caractères, mais il est possible d'étendre cet algorithme pour utiliser toutes les valeurs possibles de l'ASCII, ou même de l'Unicode.
		\subsection{Algorithme}
			Cet algorithme de chiffrement est réputé pour être très simple à mettre en \oe{uvre} et efficace à l'exécution, car le temps que prendra cet algorithme à chiffrer un caractère ne dépend ni de la longueur de la clé, ni de celle du message. La courbe représentant le temps que prendrait le chiffrement d'un message en fonction du nombre de caractères serait donc une droite, ce qui montre que cet algorithme est d'une complexité linéaire de l'ordre de O(n) où n est la longueur du message à chiffrer.\\
			Pour chiffrer un message en utilisant le système de Vigenère, il faut considérer qu'un message de longueur n est formé de caractères ayant des indices variant entre 0 et n - 1 dans un tableau. On considère la clé de la même façon. On obtient le caractère du message chiffré à l'indice i, noté C\lbrack{i}\rbrack, de la manière suivante :\linebreak
			\\
			C\lbrack{i}\rbrack = (M\lbrack{i}\rbrack + K\lbrack{i \% l}\rbrack) \% j\\
			\\
			où M\lbrack{i}\rbrack est le caractère à l'indice i du message initial M (on rappelle que l'indice varie entre 0 et n - 1), l est la longueur de la clé de chiffrement, K\lbrack{i \% l}\rbrack est le caractère de la clé de chiffrement à l'indice i modulo l, et j est le nombre de caractères codables dans le message initial. Dans le cadre de cette formule, on définit également l'opérateur \% comme l'opérateur renvoyant le reste de la division euclidienne de la partie gauche par la partie droite.\\
			L'opération peut être inversée pour obtenir le message initial à partir du message chiffré C et de la clé de chiffrement K, en utilisant la formule suivante :\\
			\\
			M\lbrack{i}\rbrack = (C\lbrack{i}\rbrack - K\lbrack{i}\rbrack + j) \% j\\
			\\
			Par exemple, si on veut transmettre le message "LES CAROTTES SONT CUITES" avec une clé (convenue à l'avance avec le destinataire) "IMPORTANT", on obtiendra :\\
			\begin{tabular}{ | c | c | c | c | c | c | c | c | c | c | c | c | }
				\hline
				Message & L&E&S&C&A&R&O&T&T&E&S \\ \hline
				Clé & I&M&P&O&R&T&A&N&T&I&M \\ \hline
				Message chiffré & T&Q&H&Q&R&K&O&G&M&M&E \\
				\hline
			\end{tabular}\\
			\begin{tabular}{ | c | c | c | c | c | c | c | c | c | c | c | }
				\hline
				Message & S&O&N&T&C&U&I&T&E&S \\ \hline
				Clé & P&O&R&T&A&N&T&I&M&P \\ \hline
				Message chiffré & H&C&E&M&C&H&B&B&Q&H \\
				\hline
			\end{tabular}\\
			Le message que l'on enverra au destinataire sera donc "TQH QRKOGMME HCEM CHBBQH", ce qui est totalement illisible pour quiconque n'aurait pas la clé de déchiffrement.
		\subsection{Niveau de sécurité}
			Dans le cas où la clé de chiffrement est de taille supérieure (inutile) ou égale à la longueur du message, et ne contient pas de répétitions, cet algorithme est réputé pour chiffrer des données de manière inviolable, c'est-à-dire qu'il est impossible de retrouver les données d'origine si on n'a aucune information sur ces dernières, car les données obtenues ressembleront à des données purement aléatoires. En pratique, cette méthode de chiffrement est employée avec une clé de taille inférieure à la taille du message. Dans ce cas-là, il est possible de pratiquer une cryptanalyse si l'on connaît la nature des données initiales, et si la clé est vraiment très petite par rapport à la taille du message à chiffrer. Pour plus d'informations, se référer à la partie sur la cryptanalyse du système de Vigenère.
		\subsection{Cas d'utilisation}
			Dans le cas courant, on considère le système de Vigenère comme permettant de coder un message dans un alphabet donné. Si l'on considère que l'alphabet en question est formé de 256 caractères, alors il est possible de coder des octets, car un octet peut prendre $2^8 = 256$ valeurs différentes. En réalité, le système de Vigenère n'est pas considéré comme suffisamment sécurisé pour chiffrer des données sensibles, car pour que ces dernières soient transmises de manière totalement sécurisée, il faudrait que la clé soit aussi longue que le message à coder, ce qui ne présente aucun intérêt. On préférera des algorithmes plus sécurisés avec des longueurs de clés moindres.
	\section{Système de César}
		\subsection{Définition}
			Le système de César est un cas particulier du système de Vigenère, où la clé est de longueur 1, c'est-à-dire que la clé utilisée pour encoder chaque caractère sera toujours la même. Il constitue également un cas particulier du système de chiffrement par substitution, où les lettres de la clé seraient écrites avec un décalage d'un nombre fixe de positions par rapport à l'alphabet normal. La clé serait donc l'alphabet utilisé pour écrire le message, décalé du nombre susdit.
			La manière dont on chiffre des données en utilisant ce système est la même que pour le système de Vigenère. Il s'agit donc d'une méthode de chiffrement symétrique.
		\subsection{Exemple}
			Imaginons-nous quelques instants à la place de Jules César. Nous souhaitons envoyer un message stratégique à nos troupes, dont nous ne voulons pas que le peuple ennemi (les gaulois) ne prennent connaissance. Ce message, c'est "ASSIEGEZ MASSILIA". Jules César employait souvent cette méthode de chiffrement avec un décalage de 3 lettres, d'après plusieurs sources. Nous allons donc faire de même.\\
			Pour rendre la tâche plus simple, on décide d'établir une table de substitution pour savoir quel caractère remplacera lequel, une fois notre message chiffré :\\
			\begin{tabular}{ | c | c | c | c | c | c | c | c | c | c | c | c | c | c | c | c | }
				\hline
				Normal&A&B&C&D&E&F&G&H&I&J&K&L&M&N&O\\ \hline
				Décalé&D&E&F&G&H&I&J&K&L&M&N&O&P&Q&R\\
				\hline
			\end{tabular}\\
			\begin{tabular}{ | c | c | c | c | c | c | c | c | c | c | c | c | }
				\hline
				Normal&P&Q&R&S&T&U&V&W&X&Y&Z\\ \hline
				Décalé&S&T&U&V&W&X&Y&Z&A&B&C\\
				\hline
			\end{tabular}\\
			\\
			Pour notre message, cela donnera donc :\\
			\begin{tabular}{ | c | c | c | c | c | c | c | c | c | c | c | c | c | c | c | c | c | }
				\hline
				Normal&A&S&S&I&E&G&E&Z&M&A&S&S&I&L&I&A \\ \hline
				Chiffré&D&V&V&L&H&J&H&C&P&D&V&V&L&O&L&D \\
				\hline
			\end{tabular}\\
			\\
			Nous enverrons alors à nos troupes le message "DVVLHJHC PDVVLOLD", qui ne veut rien dire en tant que tel, mais qui prend tout son sens une fois déchiffré par nos troupes. Par contre, il faut avoir convenu le décalage à l'avance...
		\subsection{Niveau de sécurité}
			Le niveau de sécurité de cette méthode de chiffrement est très faible, c'est-à-dire qu'il est très facile de retrouver le message d'origine, car généralement, le nombre de décalages possible est très petit (26 pour l'alphabet français). Si l'on considère que l'on ne prendra pas le décalage nul (car sinon cela reviendrait à ne pas chiffrer du tout), il ne reste que 25 possibilités pour une utilisation "classique". Il suffira alors d'écrire le message avec tous les décalages possibles, pour voir parmi la liste des 25 messages le message en clair. Par ailleurs, on peut aussi effectuer une analyse de fréquences d'apparition des caractères, et considérer par exemple, que si le message a été écrit en français, et que la lettre la plus fréquente dans un texte en français est le 'E', alors la lettre la plus fréquente du message doit être un E, et ainsi on peut déduire le décalage qui a été utilisé pour chiffrer le message.
		\subsection{Cas d'utilisation}
			L'histoire raconte que le chiffrement de César, comme son nom l'indique, a été initialement inventé par Jules César, un grand empereur de l'époque de l'antiquité. Il aurait employé cette méthode pour transmettre des messages à ses troupes, de manière à ce que même si le message était intercepté par un ennemi, ce dernier ne pourrait pas le comprendre, en raison d'abord de la langue d'origine qui était différente, mais aussi du fait que les suites de lettres inscrites sur le support formaient des mots qui n'existaient pas. Le décalage qu'il employait valait 3, c'est-à-dire que pour chaque lettre du message, il fallait regarder 3 lettres plus loin dans l'alphabet pour trouver la lettre à écrire pour chiffrer le message. Par exemple, lorsque le message à chiffrer contenait un 'A', César écrivait donc un 'D'.
			De nos jours, du fait de son niveau de sécurité, le chiffrement de César n'est plus utilisé pour chiffrer des données sensibles. En revanche, il constitue un exemple parfait pour une initiation à la cryptographie, car il demeure très simple à comprendre, et le fait qu'il soit symétrique facilite la compréhension.
	\section{Système de substitution}
		\subsection{Définition}
			Le système de chiffrement par substitution est une méthode de chiffrement qui consiste à définir, pour chaque caractère codable dans l'alphabet du message d'origine, le caractère par lequel il sera remplacé dans le message chiffré. Il faut également prendre en compte que chaque caractère ne doit être utilisé qu'une seule fois dans la clé, car sinon il pourrait y avoir ambiguïté lors du déchiffrement. Par exemple, si l'on définit que les lettres A et B seront représentées par un C dans le message chiffré, lors du déchiffrement, on ne pourra pas savoir si C était à la base un A ou un B dans le message initial.
		\subsection{Algorithme}
			L'algorithme en soi est relativement simple, il se résume à faire une passe sur chaque caractère du message et à le transformer par le caractère correspondant dans la table de substitution. Pour expliquer cela plus simplement, prenons M[i] le caractère du message initial M à l'indice i, C[i] le caractère du message chiffré à l'indice i, et K[a] le caractère de substitution du caractère a dans la table de substitution K.\\
			L'obtention du message chiffré se fait donc en une boucle, et on obtient chaque caractère de C ainsi :\\
			$C[i] = K[M[i]]$\\
			\\
			Prenons l'exemple d'un message que nous souhaitons transmettre de manière sécurisée en passant par des tiers qui connaissent la cryptanalyse des systèmes de César et de Vigenère. Ce message est : "RAPATRIEMENT".\\
			\\
			On commence alors par établir une table de substitution, dans laquelle on va définir par quel lettre chaque lettre sera remplacée : \\
			\begin{tabular}{ | c | c | c | c | c | c | c | c | c | c | c | c | c | c | c | c | }
				\hline
				Normal&A&B&C&D&E&F&G&H&I&J&K&L&M&N&O\\ \hline
				Substitué&W&X&C&V&B&N&Q&S&D&F&G&H&J&K&L\\
				\hline
			\end{tabular}\\
			\begin{tabular}{ | c | c | c | c | c | c | c | c | c | c | c | c | }
				\hline
				Normal&P&Q&R&S&T&U&V&W&X&Y&Z\\ \hline
				Substitué&M&A&Z&E&R&T&Y&U&I&O&P\\
				\hline
			\end{tabular}\\
			\\
			Une fois cette table créée, on va remplacer les lettres du messages qui se trouvent dans la première ligne du tableau par les lettres qui se trouvent à la deuxième ligne du tableau.\\
			Le message final sera alors "ZWMWRZDBJBKR". Il est totalement illisible tel quel, et il est assez petit pour qu'une cryptanalyse ne fonctionne pas dessus.
		\subsection{Niveau de sécurité}
			Cet algorithme est d'un niveau de sécurité très faible en général, et surtout si l'on connaît la nature des données chiffrées. Le chiffrement RSA utilisé naïvement est d'ailleurs un système de chiffrement par substitution, car pour un même caractère codé, le caractère chiffré sera le même. Le système de César en est également un cas particulier.
		\subsection{Cas d'utilisation}
			En raison de son faible niveau de sécurité, le chiffrement par substitution n'est plus utilisé de nos jours. Néanmoins, il existe une variante qui est le système de chiffrement par substitution polyalphabétique, c'est-à-dire que la clé n'est non plus une table de substitution, mais une liste de tables de substitution. Ainsi, il est possible de coder le même caractère deux fois de suite différemment. Là encore, le système de Vigenère se trouve en être un cas particulier, où les tables de substitutions de la clé sont formées à partir d'un décalage.