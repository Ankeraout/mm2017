\chapter{Le chiffrement RSA}
	\section{L'algorithme de chiffrement RSA}
		RSA est un algorithme de chiffrement qui nécessite une mise en place avant de pouvoir échanger des messages. Du fait qu'il s'agisse d'un algorithme de chiffrement asymétrique, les deux parties doivent faire ce travail de mise en place pour créer deux canaux de communication (l'un ascendant, l'autre descendant), afin que les deux parties puissent échanger entre elles, car sinon il s'agirait d'une communication à sens unique.\\
		\\
		Pour initialiser le chiffrement RSA, il faut commencer par générer une clé privée. Pour cela, on prend deux nombres premiers (de préférence très grands), strictement supérieurs à 2, qu'on appelle $p$ et $q$.\\
		\\
		On calcule ensuite leur produit $n = p \times q$. On appelle ce nombre le module de chiffrement. Il constitue la moitié des clés de chiffrement.\\
		\\
		On prend ensuite $\phi{n} = (p - 1) \times (q - 1)$.\\
		Maintenant que nous avons fait ces calculs, nous pouvons chercher l'exposant de chiffrement (appelé $e$), qui soit un nombre premier avec $\phi{n}$ et strictement inférieur à celui-ci. De plus, ce nombre doit être strictement supérieur à $p$ et à $q$. Ce dernier constitue la deuxième moitié de la clé de chiffrement, qu'on appellera clé publique, puisque c'est celle que l'on donnera à l'autre partie afin qu'elle puisse chiffrer son message. Il existe généralement plusieurs possibilités, il faut en choisir une au hasard (de préférence).\\
		\\
		On calcule enfin $d = \frac{1}{e} \% \phi{n}$. Il constitue la deuxième moitié de la clé de déchiffrement, que nous appellerons clé privée, car c'est celle que la partie doit conserver secrètement afin de déchiffrer les messages envoyés par l'autre partie en utilisant la clé publique qu'on lui aura préalablement envoyé.\\
		\\
		Maintenant que nous avons terminé l'initialisation, nous pouvons transmettre le couple (n; e) à la partie qui souhaite envoyer un message (ce couple constituant la clé publique), et conserver précieusement le couple (n; d), qui constitue la clé privée.\\
		\\
		Pour chiffrer un message, la partie souhaitant envoyer le message doit faire un calcul. À l'aide de l'algorithme RSA, nous pouvons envoyer des nombres. Soit $M$ le message (le nombre à envoyer), $e$ et $n$ les composants du couple (n; e).\\
		\\
		Le résultat de l'opération de chiffrement de $M$ est appelé $C$, et il vaut\\
		\\
		$C = M^e \% n$.\\
		\\
		Ce nombre peut donc nous être envoyé, et nous pouvons retrouver le nombre initial à l'aide du couple (n; d) de cette manière :\\
		\\
		$M = C^d \% n$\\
		\\
		On peut, en utilisant cette méthode, envoyer des séries de nombres. Dans la réalité, on transmet des octets via RSA. Or les octets ne peuvent prendre que des valeurs allant de 0 à 255 (inclus).\\
		\\
		Si l'on chiffre un octet en utilisant cette méthode, il n'y aura donc que 256 possibilités, et le chiffrement RSA reviendrait alors à un chiffrement par substitution. Il serait alors facile de retrouver le message initial, par cryptanalyse.\\
		Dans la réalité, le chiffrement de données par RSA n'est pas réalisé de cette manière (que l'on peut qualifier de naïve).\\
		\\
		Pour chiffrer un message, il faut tout d'abord le découper en plusieurs blocs supérieur à 2.
		\\\\\textbf{Exemple}\\
		Pour le mot "RSA", convertissons chaque lettres par sa place dans l’alphabet : "RSA" deviens donc "18 19 00". Or le découpage n'est que de deux pour chaque bloc, réajustons le résultat obtenu pour obtenir des blocs de 3, ce qui donne : "181 900". Il suffit ensuite de transformer chaque blocs non chiffrés \textbf{B} en blocs chiffrés \textbf{C} en utilisant la formule \textbf{C = B\up{e} mod n}. Le message chiffré sera le résultat de tout les blocs chiffrés. 
	\section{Niveau de sécurité}
		Tout l'intérêt du système RSA repose sur le fait qu'à l'heure actuelle il est pratiquement impossible de retrouver dans un temps raisonnable p et q à partir de n si celui-ci est très grand. La personne qui détient la clé privé est donc la seule à pouvoir déchiffrer le message dans un temps raisonnable. De plus, elle n'a jamais à transmettre les entiers p et q, ce qui empêche leur piratage.
		
		
	\section{Efficacité}
		Une des grandes forces du RSA et qu'il est encore bien protégé à l'heure actuelle (en utilisant à minima 2048 bits). Cependant ce système de chiffrement n'est pas une fin en soit. Il reste tout de même très long d'exécution et ne permet donc pas le chiffrement de très grands messages.\\
		
		Exemple du temps d'exécution de l'algorithme RSA-2048\\
			Test effectué avec :\\
			- Processeur : Intel(R) Core(TM) 2 CPU T5200 1,6GHZ.\\
			- RAM : 1,00 Go.\\
			
			Pour un fichier de 2,58Mo :\\
				- 21621ms de temps d'exécution\\\\ 
			Pour un fichier de 9,16Mo :\\
				- 74583ms de temps d'exécution\\
				
		
	\section{Cas d'utilisation}
		Le chiffrement RSA est maintenant utilisé dans quasiment tout les systèmes qui nécessitent le chiffrement de données. On peut par exemple retrouver son utilisation dans le système bancaire, le web, les cartes à puces, etc..
		Comme dit précédemment, le RSA étant bien trop coûteux en terme de temps de chiffrement, il est principalement utilisé pour chiffrer des clés de chiffrements symétrique très grande et ce sont ces clés qui serviront à chiffrer l'information. Toute fois elle