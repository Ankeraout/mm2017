\chapter{Le chiffrement RSA}
	\section{L'algorithme de chiffrement RSA}
		RSA est un algorithme de chiffrement qui nécessite une mise en place avant de pouvoir échanger des messages. Du fait qu'il s'agisse d'un algorithme de chiffrement asymétrique, les deux parties doivent faire ce travail de mise en place pour créer deux canaux de communication (l'un ascendant, l'autre descendant), afin que les deux parties puissent échanger entre elles, car sinon il s'agirait d'une communication à sens unique.\\
		\\
		Pour initialiser le chiffrement RSA, il faut commencer par générer une clé privée. Pour cela, on prend deux nombres premiers (de préférence très grands), strictement supérieurs à 2, qu'on appelle $p$ et $q$.\\
		\\
		On calcule ensuite leur produit $n = p \times q$. On appelle ce nombre le module de chiffrement. Il constitue la moitié des clés de chiffrement.\\
		\\
		On prend ensuite $\phi{n} = (p - 1) \times (q - 1)$.\\
		Maintenant que nous avons fait ces calculs, nous pouvons chercher l'exposant de chiffrement (appelé $e$), qui soit un nombre premier avec $\phi{n}$ et strictement inférieur à celui-ci. De plus, ce nombre doit être strictement supérieur à $p$ et à $q$. Ce dernier constitue la deuxième moitié de la clé de chiffrement, qu'on appellera clé publique, puisque c'est celle que l'on donnera à l'autre partie afin qu'elle puisse chiffrer son message. Il existe généralement plusieurs possibilités, il faut en choisir une au hasard (de préférence).\\
		\\
		On calcule enfin $d = \frac{1}{e} \% \phi{n}$. Il constitue la deuxième moitié de la clé de déchiffrement, que nous appellerons clé privée, car c'est celle que la partie doit conserver secrètement afin de déchiffrer les messages envoyés par l'autre partie en utilisant la clé publique qu'on lui aura préalablement envoyé.\\
		\\
		Maintenant que nous avons terminé l'initialisation, nous pouvons transmettre le couple (n; e) à la partie qui souhaite envoyer un message (ce couple constituant la clé publique), et conserver précieusement le couple (n; d), qui constitue la clé privée.\\
		\\
		Pour chiffrer un message, la partie souhaitant envoyer le message doit faire un calcul. À l'aide de l'algorithme RSA, nous pouvons envoyer des nombres. Soit $M$ le message (le nombre à envoyer), $e$ et $n$ les composants du couple (n; e).\\
		\\
		Le résultat de l'opération de chiffrement de $M$ est appelé $C$, et il vaut\\
		\\
		$C = M^e \% n$.\\
		\\
		Ce nombre peut donc nous être envoyé, et nous pouvons retrouver le nombre initial à l'aide du couple (n; d) de cette manière :\\
		\\
		$M = C^d \% n$\\
		\\
		On peut, en utilisant cette méthode, envoyer des séries de nombres. Dans la réalité, on transmet des octets via RSA. Or les octets ne peuvent prendre que des valeurs allant de 0 à 255 (inclus).\\
		\\
		Si l'on chiffre un octet en utilisant cette méthode, il n'y aura donc que 256 possibilités, et le chiffrement RSA reviendrait alors à un chiffrement par substitution. Il serait alors facile de retrouver le message initial, par cryptanalyse.\\
		Dans la réalité, le chiffrement de données par RSA n'est pas réalisé de cette manière (que l'on peut qualifier de naïve).\\
		\\
		Pour chiffrer un message à l'aide de RSA, il faut le découper en "blocs".\\
		Ces blocs sont des nombres strictement inférieurs à n.\\
		Pour les constituer, il faut écrire le message à envoyer sous la forme d'une série de nombres, et on découpe cette série de nombres en un ensemble de blocs inférieurs à $n$. Ceci étant assez difficile à expliquer, voici un exemple :\\
		Imaginons qu'on ait un chiffrement RSA avec n=283189.\\
		Imaginons également que l'on veuille envoyer le message "MESSAGE". Pour cela, on va l'écrire sous la forme d'une série de nombres, en convertissant chaque caractère du message en sa valeur ASCII. Comme ASCII est un codage qui emploie des nombres allant jusqu'à 3 chiffres en base 10, on écrira les nombres obtenus sur 3 chiffres. On obtiendra alors quelque chose de ce genre :\\
		077 069 083 083 065 071 069\\
		\\
		On crée, à partir de cette série de chiffres, une série de nombres "blocs", strictement inférieurs à $n$ :\\
		077069 083083 065071 069000\\
		\\
		Ce sont ces nombres que l'on chiffrera grâce à l'algorithme RSA.\\
		\\
		Lors du déchiffrement, ces nombres seront reconvertis en valeurs numériques en prenant des blocs de 3 chiffres par 3, pour reconstituer ensuite les valeurs ASCII et retrouver le message original.\\
		C'est grâce à ce processus que le message est sécurisé lors de sa transmission, car le chiffrement ne revient alors plus à un chiffrement par substitution, et il y a alors n blocs codables, ce qui fait que l'on utilisera pleinement les possibilités qu'offre RSA.\\
		Pour chiffrer un message, il faut tout d'abord le découper en plusieurs blocs supérieur à 2.
	\section{Niveau de sécurité}
		RSA étant un algorithme de chiffrement asymétrique, la clé permettant de déchiffrer un message chiffré en utilisant cet algorithme (clé privée) n'est jamais transmise. Or il est impossible de déchiffrer un message intercepté si on ne possède pas la clé privée de la personne qui était supposée recevoir ce message (et le déchiffrer). C'est grâce à cela que RSA est un algorithme particulièrement réputé par sa sécurité.\\
		RSA n'est pourtant pas un algorithme invulnérable, car pour le casser, il "suffit" de factoriser la valeur $n$, commune aux clés privées et publiques d'un échange RSA.\\
		En factorisant $n$, le tiers ayant intercepté le message devient capable de reconstituer la clé privée permettant de déchiffrer le message.\\
		Seulement, la factorisation de $n$ est un processus excessivement compliqué si les nombres $p$ et $q$ sont des nombres premiers très grands.\\
		À l'heure actuelle, en raison de la relative faible puissance de nos ordinateurs, il est très difficile de factoriser un très grand nombre composé si ses constituants sont de très grands nombres premiers.
	\section{Efficacité}
		Une des grandes forces du RSA et qu'il est encore bien protégé à l'heure actuelle (en utilisant à minima 2048 bits). Cependant ce système de chiffrement n'est pas une fin en soi. Il possède plusieurs désavantages, notamment sa lenteur et sa consommation de mémoire, qui augmentent avec la grandeur des nombres qu'il manipule. Plus on avance dans le temps, et plus cet algorithme devient donc consommateur de ressources. Il permet malgré tout le chiffrement de messages de manière très fiable, et en général il est plutôt utilisé pour sécurisé l'échange de clés pour des algorithmes de chiffrement symétriques, beaucoup plus rapides et moins gourmands en terme de ressources système.\\
		\\
		Pour se rendre compte de l'impact de cet algorithme sur les performances, voici un test de chiffrement de fichiers en utilisant RSA avec une clé de 2048 bits sur un ordinateur bas/moyen de gamme :\\
			Test effectué avec :\\
			- Processeur : Intel(R) Core(TM) 2 CPU T5200 1,6GHZ.\\
			- RAM : 1,00 Go.\\
			
			Pour un fichier de 2,58Mo :\\
				- 21621ms de temps d'exécution\\\\ 
			Pour un fichier de 9,16Mo :\\
				- 74583ms de temps d'exécution\\
	\section{Cas d'utilisation}
		Le chiffrement RSA est maintenant utilisé dans quasiment tout les systèmes qui nécessitent le chiffrement de données. On peut par exemple retrouver son utilisation dans le système bancaire, le web, les cartes à puces, etc..
		Comme dit précédemment, le RSA étant bien trop coûteux en terme de temps de chiffrement, il est principalement utilisé pour chiffrer des clés pour des algorithmes de chiffrement symétrique et ce sont ces derniers qui serviront à chiffrer l'information.\\
		\\
		Parmi les utilisations les plus courantes de cet algorithme, il y a également les certificats.\\
		Pour résumer, un certificat est un fichier permettant d'identifier de manière sûre une entité. Ils sont envoyés à chaque visite d'un site web, afin de s'assurer que la personne qui nous envoie la page est bien le site que l'on demande, grâce à la complicité d'un tiers de confiance (une autre entité validant l'identité du site web).\\
		\\
		Le certificat comporte au minimum une signature du site, chiffrée grâce à sa clé privée (et non publique), et déchiffrable par tout le monde grâce à une clé publique, elle aussi contenue dans le certificat. Cela permet de s'assurer que la signature contenue dans le certificat a bien été créée par la bonne entité, car sa modification est impossible étant donné que seule la clé privée permettrait cela, et elle n'est pas distribuée.\\
		\\
		Dans le cadre d'une communication Web, le protocole utilisé est TLS, et ce dernier exige un certificat afin de créer le tunnel de données sécurisé.