\chapter{Cryptanalyse}
	\section{Définition}
		La cryptanalyse est la science qui consiste à déchiffrer un message chiffré, c’est-à-dire tenter de déchiffrer ce message sans posséder la clé de chiffrement. Le processus par lequel on tente de comprendre un message en particulier est appelé une attaque.
	\section{Aparté : Alice et Bob}
		Lorsqu'on explique par la pratique un algorithme de chiffrement, on l'explique en utilisant l'exemple de deux personnes, Alice et Bob, Alice souhaitant envoyer un message à Bob. Pour l'anecdote, ces prénoms ont été choisis par Ron Rivest, dans un article présentant l'algorithme RSA en 1978. Il a choisi de donner des prénoms à ses personnages afin que son explication soit plus claire que s'il avait employé des noms génériques (personne A et personne B, par exemple).\\
		Dans les exemples que nous allons présenter dans cette partie, nous prendrons donc cet exemple.\\
		Imaginons deux élèves à l'école, Alice et Bob. Alice souhaite discuter avec Bob en classe, via un moyen de communication rudimentaire mais néanmoins fonctionnel : le passage d'un morceau de papier d'élève en élève avec un message écrit dessus. Le problème, c'est que le message qu'Alice souhaite transmettre est strictement confidentiel, et que chaque élève peut décider d'ouvrir le papier et d'en lire le contenu. Alice doit donc trouver avec Bob un moyen de rendre le message inintelligible par ses camarades de classe.\\
		Le seul problème, c'est que Oscar, l'un de leurs camarades, se trouve être le meilleur élève de la classe, et il connaît la plupart des stratagèmes que pourraient utiliser ses camarades. Nous allons donc voir comment il fait pour réussir malgré tout à lire le message.
	\section{Cryptanalyse du système de César}
		Le système de César est un cas particulier du système de Vigenère, très facile à attaquer si l’on connaît la langue du texte d’origine. Il suffit d’analyser les fréquences, et considérer que la lettre qui revient le plus fréquemment est la lettre la plus fréquente dans la langue du message. Par exemple en français, on va considérer que la lettre qui revient le plus souvent est un ‘E’. En connaissant ceci, il est facile de retrouver le décalage qui a servi à chiffrer le message.\\
		\\
		Voici un exemple concret :\\
		Imaginons que Alice ait envoyé le message suivant à Bob :
		\begin{lstlisting}
OH SHUVRQQDJH GX OLYUH V'DSSHOOH EHQRLW.
		\end{lstlisting}
		Oscar sait que la lettre que l'on retrouve le plus fréquemment dans un texte écrit en français est la lettre E.
		Il va donc calculer les fréquences d'apparition des lettres dans le message :\\
		\begin{tabular}{ | c | c | c | c | c | c | c | c | c | c | c | c | c | c | c | c | }
			\hline
			Lettre & A & B & C & D & E & F & G & H & I & J & K & L & M & N & O\\ \hline
			Fréquence & 0 & 0 & 0 & 2 & 1 & 0 & 1 & 7 & 0 & 1  & 0 & 2 & 0 & 0 & 4\\
			\hline
		\end{tabular}\\
		\\
		\begin{tabular}{ | c | c | c | c | c | c | c | c | c | c | c | c | }
			\hline
			Lettre & P & Q & R & S & T & U & V & W & X & Y & Z\\ \hline
			Fréquence & 0 & 3 & 2 & 3 & 0 & 2 & 2 & 1 & 1 & 1 & 0\\
			\hline
		\end{tabular}\\
		\\
		Oscar remarque alors que la lettre qui revient le plus fréquemment est la lettre H. Il va donc regarder le décalage entre la lettre E et la lettre H. Il s'agit d'un décalage de 3 positions.\\
		Oscar se met donc à réécrire le message, en prenant pour chaque lettre, la lettre située 3 positions avant dans l'alphabet. S'il tombe sur une lettre dont la position est inférieure ou égale au décalage calculé, il revient à la lettre Z (par exemple, la lettre B donnera Y).\\
		\\
		À la fin de son opération, Oscar trouve un message écrit en français, parfaitement intelligible :
		\begin{lstlisting}
LE PERSONNAGE DU LIVRE S'APPELLE BENOIT
		\end{lstlisting}
		Oscar a donc réussi à casser facilement le chiffrement que Alice avait appliqué à son message.
	\section{Cryptanalyse du système de Vigenère}
		Afin de décrypter un message chiffré en utilisant le système de Vigenère, il faut tout d’abord déterminer la longueur de la clé qui a servi à chiffrer le message. Pour cela, on utilise un outil appelé l’indice de coïncidence.
		Il s’agit d’un outil de cryptanalyse, inventé par William F. Friedman, un cryptologue de l’armée de terre américaine en 1920. Il représente la probabilité, pour un texte donné, que deux lettres prises au hasard dans ce texte soient les mêmes.
		En français, il vaut environ 0.746 (ou 0.778, selon les sources).
		Lors de notre cryptanalyse, on va tester tous les découpages de textes avec un espacement entre les caractères variant dans l’intervalle [m; n] où m et n sont des entiers naturels différents de zéro (m >= n), minorant et majorant la taille de la clé de chiffrement. Pour chaque découpage, on calcule la moyenne des indices de coïncidence de tous les textes, et on retient, pour chaque valeur de découpage, l’indice de coïncidence. La longueur probable de la clé est donc le découpage qui donne un indice de coïncidence le plus proche de l’indice de coïncidence du texte donné.\\
		\\
		Prenons en exemple le texte suivant :
		\begin{lstlisting}
cbntjdefhccavtvfdvfbpufdftgnzqwkysdufwcnfjguznzutvtlveony
byvldcfupipfovtxgdglxfdcdgfibyvncywjogyhtrmtvpofkippewnzu
pgewykfrtthpvodctrleudqfjpeuogdsnifeuzincfdgwzpjfokdhtehf
goswrtegroyfhccavtvrfkqotkqludsczolrpfnlvyopgdrhpflbdlolw
efpdfduluprmztdefvmlecmakprtldszfcgyrfenpudorvjykyhpcmtit
pwvblwefphvpsfwovecqthpcmpgdhfkjwkdspufawtgwroekbitkfxctg
nvsectbpjepupgxvusqosdcfdrwidznaqchleupuncxdfwcnfjguzicoa
yjpcdmxvuckbipubegyhovmlhtbolwtprhtvnputsncf
		\end{lstlisting}
		C'est le message que Alice souhaite envoyer à Bob, mais que Oscar a intercepté. Il sait que le texte est écrit en français, et puisqu'il n'arrive pas à un résultat concluant à l'aide d'une cryptanalyse du système de César, il décide de tenter une cryptanalyse pour un système de Vigenère.\\
		\\
		Tout d'abord, Oscar va découper le texte $T$ en $n$ sous-textes $s_i$ où $i$ est le numéro du sous-texte (partant de zéro) en prenant dans $s_i$ les lettres du texte initial dont la position $p$ (partant de zéro) est une solution de l'équation $p \% n = i$, c'est-à-dire que si on découpe $T$ en 3 parties, on mettra dans $s_1$ la deuxième lettre de $T$, puis la 5$^{eme}$ etc, car $1 \% 3 = 1$, $4 \% 3 = 1$, etc.\\
		\\
		Une fois ce découpage fait, il va calculer l'indice de coïncidence de chacun des sous-textes et en faire la moyenne. Il va ensuite réitérer l'opération plusieurs fois afin de tester plusieurs découpages, afin de constituer un tableau des indices de coïncidence en fonction des découpages. Voici ce que Oscar obtient pour ce texte en ayant testé les découpages de 2 à 10 :\\
		\begin{tabular}{ | c | c | c | c | c | c | c | c | }
			\hline
Découpage (n) & 2 & 3 & 4 & 5 & 6 & 7 & 8 \\ \hline
Indice de coïncidence & 0.045 & 0.044 & 0.043 & 0.044 & 0.044 & 0.080 & 0.043 \\
			\hline
		\end{tabular} \\
		\\
		Oscar remarque que l'indice de coïncidence pour le découpage en 7 textes est très proche de celui du français (0.075 environ) comparé aux autres. Il en déduit que la clé doit être de longueur 7.\\
		\\
		Maintenant qu'il connaît la longueur de la clé, il conserve ses sous-textes et effectue une cryptanalyse de César dessus pour retrouver chacun des caractères de la clé. À la fin de ce processus, il obtient la clé "rblczdl".\\
		En déchiffrant le message en utilisant cette clé, Oscar a un texte partiellement illisible :
		\begin{lstlisting}
lacrkatograbsieestgyedesdudciplizpsdelaocyptolariesatflchant
marotegqcdesmeedagesaedurantoznfidezeialitqluthenftciteeftnt
egrueeensauoantsoggentdeepcretsafclesexwesedieeinguepplastes
lnograbsiequirlitpaseprinapqccuunmqdsagedmysunaufcemessmreal
orebuelacdjptogrmahierezounmeselgeinizeelligumleaaufcequeqgt
dedroueelleeeeutiliepedepuudlanticfitemaudcertauyesdesqdmeth
oppslespxfsimpodeantesozmmelaocyptogdlphieaejmetricfedatezed
elafuyduvinseiemesupcle
		\end{lstlisting}
		Il reconnaît pourtant au début du texte "La cryptographie". En modifiant un peu sa clé pour retrouver "La cryptographie" au début du texte (la nouvelle clé étant rblclol), Oscar arrive à avoir ce texte :
		\begin{lstlisting}
lacryptographieestunedesdisciplinesdelacryptologiesattachant
aprotegerdesmessagesassurantconfidentialiteauthenticiteetint
egriteensaidantsouventdesecretsouclesellesedistinguedelasteg
anographiequifaitpasserinapercuunmessagedansunautremessageal
orsquelacryptographierendunmessageinintelligibleaautrequequi
dedroitelleestutiliseedepuislantiquitemaiscertainesdesesmeth
odeslesplusimportantescommelacryptographieasymetriquedatentd
elafinduvingtiemesiecle
		\end{lstlisting}
		Oscar arrive donc à lire ce texte, qui, après quelque mise en forme, est rendu parfaitement lisible :
		"La cryptographie est une des disciplines de la cryptologie s’attachant à protéger des messages, assurant confidentialité, authenticité et intégrité en s’aidant souvent de secrets, ou clés.
		Elle se distingue de la stéganographie qui fait passer inaperçu un message dans un autre message, alors que la cryptographie rend un message inintelligible à autre que qui-de-droit.
		Elle est utilisée depuis l’antiquité, mais certaines des méthodes les plus importantes, comme la cryptographie asymétrique, datent de la fin du vingtième siècle."\\
		\\
		Oscar a donc réussi à déchiffrer le message d'Alice, car le texte était suffisamment long pour se prêter à l'expérience, mais aussi parce qu'il connaissait la langue d'origine du texte.
	\section{Cryptanalyse du système de substitution}
		Pour décrypter un message chiffré par un système de substitution, on va effectuer une analyse fréquentielle des lettres de ce texte.\\
		Là encore, pour que cette méthode fonctionne, il faut que le texte ait été écrit dans une langue qui est connue par l'attaquant, mais il faut aussi et surtout que le texte soit long (voire très long), car durant nos recherches, nous avons remarqué que pour qu'une cryptanalyse pour un système de substitution fonctionne bien, il faut que le texte soit bien plus long que pour une cryptanalyse pour un système de Vigenère. Cela vient principalement du fait que pour le système de Vigenère, on emploie des clés de taille réduite (8 caractères ou moins), là où pour le système de substitution, dans la langue française, la clé est de taille 26 quoi qu'il arrive.\\
		\\
		Voici un extrait du texte qui a été donné par Alice :
		\begin{lstlisting}
XY AXJ BYRJMYJ, MQQMVUVXYJ GXR NCBWJR N'UYX LMBY N'PCLLX
XJ BGR XAVBDBVXYJ, XY IMAX NU AMYNXGMFVX, RUV GX QGMJVX
NU LUV NU QMGMBR VCEMG. GX VCB DBJ AXJJX QMVJBX NX LMBY
KUB XAVBDMBJ. MGCVR GX VCB APMYWXM NX ACUGXUV, RXR
QXYRXXR G'XIIVMEXVXYJ, GXR SCBYJUVXR NX RXR VXBYR RX
NXGBXVXYJ XJ RXR WXYCUZ RX PXUVJXVXYJ G'UY G'MUJVX. GX
VCB AVBM MDXA ICVAX QCUV IMBVX DXYBV GXR LMWBABXYR, GXR
APMGNXXYR XJ GXR MRJVCGCWUXR.
		\end{lstlisting}
		En voyant ce texte, Oscar a commencé par calculer le nombre d'apparitions de chaque lettre, dont voici un extrait :\\
		\begin{tabular}{ | c | c | c | c | c | c | }
			\hline
			Lettre & A & B & C & D & E\\ \hline
			Apparitions & 43 & 99 & 80 & 17 & 8\\
			\hline
		\end{tabular}\\
		\\
		Oscar retrouve ensuite un tableau de la fréquence d'apparition des lettres dans un texte français. Il trouve alors que la lettre qui apparaît le plus souvent est la lettre 'E'.\\
		Il cherche donc dans le texte quelle lettre apparaît le plus souvent, et il s'agit du 'X', qui apparaît 242 fois. Oscar peut alors considérer que les 'E' du texte initial ont été remplacés par des 'X'. Il réitère l'opération 26 fois, afin d'obtenir un tableau de substitution dont voici un extrait :\\
		\begin{tabular}{ | c | c | c | c | c | c | }
			\hline
			Lettre initiale & A & B & C & D & E \\ \hline
			Lettre de remplacement & C & I & L & G & X\\
			\hline
		\end{tabular}\\
		\\
		À la fin, en modifiant les lettres du texte grâce au tableau qu'il a construit, Oscar obtient ce texte :
		\begin{lstlisting}
en cer insrtnr, tpptaoaenr ues dliqrs d'one mtin
d'flmme er ius ecaigiaenr, en vtce do ctndeutbae, soa
ue putrae do moa do ptutis alxtu. ue ali gir cerre
ptarie de mtin hoi ecaigtir. tulas ue ali cftnqet de
cloueoa, ses pensees u'evvatxeaenr, ues jlinroaes de
ses aeins se deuieaenr er ses qenloy se feoareaenr u'on
u'torae. ue ali cait tgec vlace ploa vtiae genia ues
mtqiciens, ues cftudeens er ues tsralulqoes.
		\end{lstlisting}
		Il ne s'agit toujours pas d'un texte en français, mais Oscar remarque que certains mots ressemblent beaucoup à d'autres mots de la langue française, comme par exemple, "mtqiciens" ressemble à "magiciens". Oscar change donc à nouveau son tableau pour prendre en compte ces quelques modifications, et il arrive à la fin à un texte parfaitement lisible :
		\begin{lstlisting}
en cet instant, apparurent les doigts d'une main
d'homme et ils ecrivirent, en face du candelabre, sur
le platre du mur du palais royal. Le roi vit cette
partie de main qui ecrivait. Alors le roi changea de
couleur, ses pensees l'effrayerent, les jointures de
ses reins se delierent et ses genoux se heurterent l'un
l'autre. Le roi cria avec force pour faire venir les
magiciens, les chaldeens et les astrologues.
		\end{lstlisting}
		Le chiffrement par substitution appliqué sur un texte en français n'est donc pas suffisamment sécurisé, puisque Oscar a réussi à décrypter le texte envoyé par Alice.